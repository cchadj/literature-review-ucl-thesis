\documentclass[]{article}

\usepackage{subcaption}
\usepackage{amssymb,amsmath,amsthm}
\usepackage{svg}
\usepackage{caption, subcaption}
\usepackage[backend=bibtex]{biblatex}  
\usepackage{todonotes}

\title{Literature Review on Cell tracking across videos from different points of view}
\author{Chrysostomos Chadjiminas}


\addbibresource{biblio.bib} 

\begin{document}

\maketitle

\begin{abstract}
	This is a sample content
\end{abstract}


\section{Problem statement}

\subsection{Apo to de Castro}

A number of methods are available for measuring blood velocity. To date techniques that are based on Doppler velocimetry velocimetry [3, 4] including variations based on Optical Coherence Tomography (OCT) [5, 6] have been improved and demonstrated by many groups and have been most useful for relatively large retinal vessels (> 50 microns). Also blood flow can be related to the modulation of the speckle signal [7] but this technique tends to be useful only for large vessels as well. Direct imaging techniques have typically used 

either dye injection [8–12] or externally labeled leukocytes re-injected into the blood stream [12, 13]. Currently Adaptive Optics techniques (AO) are able to provide higher resolution images of flow in small vessels without the need of contrast agents. Using an Adaptive Optics confocal Scanning Laser Ophthalmoscope (AOSLO) the velocity and pulsatility of the large moving features thought to be leukocytes [14], that can be observed moving through the capillaries of the macular region was determined either from the change in position of two consecutive frames of a video [14, 15] or from the change of intensity in a spatiotemporal plot [16, 17]. Also, AO has allowed the measurement of the erythrocyte velocities in small retinal regions either by using high frame rate non-confocal imaging [18], or by shrinking temporarily the field of view to a single scan line [19].
Recently the use of an AOSLO with multiply scattered light detection [20, 21] has allowed improved imaging of the vasculature of the human retina. With this technique rbcs can be directly imaged while travelling through the capillaries of the human eye [22, 23]. The determination of rbcs velocity is desirable since they should provide a better estimate of tissue perfusion and vascular autoregulation than leukocytes, which are much larger than the capillary lumen and thus travel at a different velocity. While the single file flow of rbcs through capillaries simplifies velocity measures, their abundance introduces the issue of velocity aliasing for low frame rate systems, since the same cell cannot be identified between two consecutive frames. To solve this problem we developed a dual-channel AOSLO to image the retina with two close but different wavelengths. By introducing small angular shifts between the beams at the pupil, the same area of the retina is sampled at different times and this produces temporal offsets much smaller than the frame rate of the raster scan itself. Figure 1 shows the general principle of the approach. In the current study we introduced a separation of about 4.7 ms between channels and measured the change in position of the red blood cells during this interval to calculate velocity. Because the signal to noise ratio in our images is relatively low we calculated the average velocity over a 10 ms window and along a capillary segment.

\subsection{Bedggood}
% Mapping flow velocity in the human retinal capillary network Cited by 2

The retina is unique in that it affords direct, non-invasive observation of neural tissue and its associated vascular beds. 
Recent developments in technology have made possible the visualization of the smallest vessels in the living human retina [1–4]. The smallest vessels within neural tissue offer the greatest resistance to flow and are thought to play a key role in mediating functional changes in flow [5, 6].

 Such vessels have been implicated early in the disease process for a variety of conditions including diabetes [7], hypertension [8], stroke [9] and dementia [10]. Recent studies of retinal capillaries using adaptive optics imaging suggest that overt structural damage to capillaries and larger vessels may be preceded by altered capillary flow patterns [11, 12]. Proper characterisation of microvascular flow in the retina may therefore prove important to help elucidate the course of progression for a range of important diseases, and to provide a more sensitive biomarker for the evaluation of potential treatments. Whilst it is now possible to observe individual blood constituents in the smallest retinal vessels [4], including erythrocytes, leukocytes and even platelets [13], quantifying rate of flow remains challenging due to their small size, low contrast, and high velocity relative to vessel


diameter. Existing approaches include manual or part-manual labelling of data [13–15], which are time consuming and may not be amenable to assessment of large numbers of patients in clinical settings. It is possible to “freeze” the raster of a scanning system to rapidly image a given point on a vessel, as long as eye movements are tracked and compensated [13, 16]. This produces highly precise measurements of velocity and other rheological parameters at that position, at the expense of simultaneous collection of data across the vascular network. This paper will focus on automated approaches to determine velocity, from large numbers of vessels imaged simultaneously in a single video sequence. Previously advocated approaches applicable to this task include particle image velocimetry (“PIV” [4]) and the spatiotemporal kymograph (“STK” [17–21]), which suffer from various limitations summarised below. PIV involves the division of each spatially-registered image frame into small sub-regions. For each sub-region, the 2D cross-correlation between successive frames is calculated independently. The displacement of the peak of the cross-correlation indicates the distance that blood is presumed to have travelled between frames in a given sub-region. Inherent limitations include:

\subsection{Apo ton Adam Dubis}
%​The purpose of tracking blood cells is to measure their speed, that is the obvious point. In actuality, we do not know the full effect of blood pressure on single cell blood flow, that is one of the points we want to look at. Why do we want to do this? In short everytime the heart beats, a bolus of blood leaves the heart at high pressure and speed and leaves into the vascular system of the body. The blood vessels, especially the vessels attached to leaving the heart (arteries) are elastic and are able to absorb this high pressure bolus and damped the pressure as the blood gets away from the heart and into the tissues (eye for us). Physiologically the goal of the smallest blood vessels is to facilitate the exchange of O2 out of the vessels and waste products CO2 out of the tissue. From a fluid dynamics standpoint this works best if there is a constant and consistent flow of blood at a single speed. As I mentioned the arteries, when young and healthy dampen the heart beat pressure wave into a near constant flow in the retinal tissue. As the vascular system ages, or damage is done to the vascular system (diabetes, heart disease, chronic inflammatory diseases), the arterial blood vessels are not able to dampen the rate of flow and so the pressure wave tracks down the blood vessel and into the capillaries. This changes occurs long before detectable disease occurs and to some extent is likely reversible. It is believed that this loss of vessel pulse dampening is responsible the the disease effects of diabetes, dementia, Parkinson's and Multiple Sclerosis. However there is no way to study this in the brain. The vascular system of the eye and brain are similar to each other and different from everywhere else in the body. Therefore we want to study the eye. I have put two papers one by deCastro et al and one by Bedggood et al that lay out this rationale in more depth.

%) Conclusions - We have treatments for MS and diabetes that we can apply that slow progression, but we need a biomarker that shows early progression. We hope that this imaging provides. The difference between what you said is that this imaging will not provide a treatment, it hopes to enable earlier treatment with existing interventions 

\subsection{Problem statement}
In this thesis the main goal is to track and measure red blood cell velocity across frames captured within the eye retina.	
The images captured using an Adaptive Optics Light Scanning Opthalmoscope that was developed by my supervisor and him team.
From having videos of the red blood cell movement, one can deduce the velocity of the blood flow, among other characteristics.
By detecting the cells across frames, because the video is captured at a known fixed rate and the field of view is known, one can then extract the velocity of each red blood cell which is important for measuring the blood flow.

% Παράφραση from the Adam Dubis email
The retina is unique because, with development in adaptive optics, it allows for a non-invasive imaging of the vascular plexus inside the retina \cite{tam_noninvasive_2010}.
As the vascular system gets older, or damage is done to it from diabetes, heart disease, chronic inflammatory diseases or other neurological diseases, the arterial blood vessels are not able to dampen the rate of blood flow. As a result, the pressure wave tracks down the blood vessel and into the capillaries. 
These changes occur before the disease is detected. 
Contrary to the symptoms these diseases have at a later stage, these changes can be potentially reversed, effectively preventing the disease from causing more damage.  \todo{search for citations}
It is believed that this loss of vessel pulse dampening is responsible the disease effects of diabetes \cite{mizutani_accelerated_diabetes_1996}, dementia \cite{de_la_torre_is_alzheimer_2004}, stroke \cite{ostergaard_role_stroke_2013}, hypertension \cite{wolf_s_quantification_hypertension_1994}, Parkinson's and Multiple Sclerosis \cite{bateman_comparison_multiple_sclerosis_2016}.
The capillaries of the brain could serve as an effective biomarker for the existence of such diseases.
Unfortunately, examining these small vessels and the blood flow in the brain can be hard, expensive or sometimes impossible.
However, The vascular system of the eye and brain are similar to each other and different from everywhere else in the body.
As a result, the imaging of the retinal vasculature could accurate reflect the microvascular changes that happen in the brain \cite{patton_retinal_brain_vasculature_similarity_2005}. ``The eye is the window of the soul'', and more specifically, a window to the brain.

As previously mentioned, our goal is to measure the velocity of the blood cells as captured in videos of the capillaries inside the retina.
Manually detecting the blood cell's in these images and finding the correspondence is slow, cumbersome and error prone.
Therefor we want to develop a technique to detect and track the blood cells across the video to extract the blood flow velocity.
There are many challenges compacted with this 


\printbibliography
\end{document}
